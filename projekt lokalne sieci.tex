\documentclass{report}

\usepackage[a4paper, left=2.5cm, right=2cm, top=1.5cm, bottom=2cm, includefoot=false, includehead=false]{geometry}
\usepackage[MeX]{polski}
\usepackage[utf8]{inputenc}
\usepackage{enumerate}
\usepackage{graphicx}

\makeatletter

\newcommand{\linia}{\rule{\linewidth}{0.4mm}}

\renewcommand{\maketitle}{\begin{titlepage}

    \vspace*{1cm}

    \begin{center}\small

    Politechnika Wrocławska\\

    Wydział Elektroniki W-4\\

    \end{center}

    \vspace{3cm}

    \noindent\linia

    \begin{center}

      \LARGE Projekt lokalne sieci komputerowe\\
      \normalsize\textsc{\@title}

         \end{center}

     \noindent\linia

    \vspace{0.5cm}

    \begin{flushright}

    \begin{minipage}{6cm}

    \textit{\small Autor:}\\

    \normalsize \textsc{\@author} \\

    \end{minipage}

    \vspace{5cm}

     {\small Prowadzący:}\\

         Dr hab. inż. Krzysztof Walkowiak

     \end{flushright}

    \vspace*{\stretch{6}}

    \begin{center}

    \@date

    \end{center}

  \end{titlepage}

}
\makeatother

\author{Mateusz Socha 181308 \\ Janusz Kuszczyński 184872 }

\title{}

\begin{document}

\maketitle
\tableofcontents

\chapter{Wstęp}

\vspace{0,5cm}
Projekt instalacji sieciowej jest realizowany dla firmy ComputerBudy. Siedziba która jest jednocześnie przedmiotem tego projektu znajduje się
przy ulicy Szwajcarska 22 w Wrocławiu.

\begin{figure}[h!]
  \centering
      \includegraphics[width=0.8\textwidth]{./obrazki/adres_computerbudy.jpeg}
  \caption{Lokalizacja centrali firmy na mapie Wrocławia.}
\end{figure}

ComputerBudy jest firmą z działu IT. Zajmuje się ona zdalną pomocą przy problemach informatycznych. Zapewnia również zdalną administrację dla
skomplikowanych aplikacji na urządzeniach użytkownika. Jej oferta jest skierowana do osób prywatnych oraz małych i średnich firm, które
nie posiadają własnego działu IT.

Profil usług świadczonych powoduje, że brak połączenia z zewnętrzną siecią internet całkowicie paraliżuje całą firmę. 
Nawet awaria pojedynczego stanowiska powoduje straty. Restrykcyjna polityka bezpieczeństwa firmy sprawia, że nawiązanie połączenia z klientem
może nastąpić tylko z sieci firmowej. Aby zwiększyć bezpieczeństwo każde stanowisko obsługujące klientów jest przyłączone do sieci
za połączone za pomocą kabla UTP. Obostrzenia te spowodowane są obawą przed podsłuchaniem poufnych informacji przez osoby niepowołane oraz
przejęciem kontroli nad komputerem klienta podszywając się pod pracownika firmy z innej lokacji.

ComputerBudy wynajmuje łącznie 4 piętra w dwóch bliźniaczych budynkach stojących obok siebie.W pierwszym dwa i w następnym budynku kolejne dwa.
Pozostałe piętra wynajmują inne firmy.

Celem naszej pracy jest stworzenie projektu nowej instalacji teleinformatycznej na użytkowanych przez firmę piętrach w obu budynkach. Zakres projektu:
\begin{itemize}
\item{Inwentaryzacja sprzętu i infrastruktury dostępnej w przedsiębiorstwie}
\item{Analiza potrzeb użytkowników – wymagania zamawiającego}
\item{Założenia projektowe}
\item{Projekt sieci}
\subitem{Projekt logiczny sieci wraz z opisem koncepcji rozwiązania}
\subitem{Konfiguracja adresacji IP}
\subitem{Projekt okablowania}
\subitem{Projekt podłączenia do Internetu}
\subitem{Analiza bezpieczeństwa i niezawodności sieci}
\subitem{Kosztorys urządzeń}
 
\end{itemize}
Wnioskując z profilu usług firmy priorytetowe znaczenie podczas projektowania należy nadać niezawodności. Drugim w kolejności czynnikiem jest oczywiście
szeroko pojęte bezpieczeństwo. Wskazane jest również zapewnienie łatwej możliwości rozbudowy sieci w tym budynku na kolejne piętra.
Oczywiście jako, że zleceniodawca jest firmą prywatną należy zminimalizować koszty całego przedsięwzięcia.

Do stworzenia projektu instalacji teleinformatycznej zostaną użyte szczegółowe plany budynków udostępnione przez zleceniodawcę.
Wymagania użytkowników zostaną opracowane na podstawie danych przekazanych przez administratora IT firmy oraz poprzez konsultację
z samymi pracownikami. Przepustowości łącz w nowej instalacji zostaną oszacowane na podstawie danych z obecnie istniejącej sieci komputerowej.

\chapter{Inwentaryzacja sprzętu i infrastruktury dostępnej w przedsiębiorstwie}
Na podstawie udostępnionej dokumentacji oraz wizyt w budnyku mieszczącym firmę opracowano zestawienie zasobów obecnie posiadanych przez firmę.
W obecnej architekturze sieciowej razem w obu budynkach znajduje się 225 gniazdek ethernetowych. Całe obecne okablowanie wykonane jest 
za pomocą nieekranowanej skętki w kategorii 3. Jest to wyraźnie przestarzała technologia. W centrali znajduje się serwer realizujący usługę
bazy danych, serwera FTP oraz hostujący stronę internetową firmy. Serwer działa pod kontrolą systemu NetWare.

Wszystkie komputery PC oraz inne urządzenia przyłączone do sieci posiadają interfejsy sieciowe ethernet i spełaniają wymagania niezbędne
do połączenia do nowej sieci. Spis programów używanych w firmie:
\begin{enumerate}
 \item system operacyjny Windows XP
 \item przeglądarka Firefox
 \item program pocztowy Thundebird
 \item skype dla firm
 \item edytor tekstu Microsoft Office
 \item klient NetWare
 \item ssh
 \item TeamViever
 \item program księgowo kadrowy Płatnik
\end{enumerate}

Dodatkowo każdy dział ma dodatkowe oprogramowanie specyficzne dla funkcji jaką pełni. Specyfikacja ta zostanie sprecyzowana w następnym rozdziale.

Budynki w których firma ma swoją siedzibę to nowoczesne biurowce. Wynajmujący piętro sam zagospodarowuje większość 
znajdującej się tam przestrzeni za pomocą modułowej architektury boxów. Aby umożliwić dużą elestyczność konfiguracji przestrzennej piętra 
wyposażone są w podwieszane sufity w których poprowadzono jest większość instalacji. Właśnie pod kątem tego montażu zostanie zaprojektowany plan
okablowania.

Sieć energetyczna zainstalowana w budynku spełnia wszelkie wymagania dotyczące bezpieczeństwa oraz wydajności wymaganej dla sieci komputerowej.
Warte odnotowania jest obeność instalacji piorunochronowej na obu budynkach. Znacząco zwiększa to bezpieczeństow sprzętów elektronicznych zainstalowanych
w budynku.

Zakłócenia elektomagnetyczne w budyku są na tyle małe, że można je pominąć. W okolicy nie pracuje żaden duży zakaład przemysłowy, który mógłby 
znacząco wpłynąć na parametry zasilania w sieci. Inne firmy, które prowadzą swoją działalność w tych budynkach korzystają jedynie z standardowego
sprzętu birurowego połączonego kablową siecią ethernetową. Brak innych sieci bezprzewodowych w budynkach znacząco ułatwia implementacje sieci wifi
ponieważ nie występuje problem interferencji międzykanałowych.

%Trzba dodać orientacyjne oszacowanie ilości sprzętu.

%Z umieszczeniem tych planów to trzeba się jeszcze zastanowić.
\pagebreak[4]
Plany budynków znajdują się w załączniku A. Wzajemne rozmieszczenie budynków przezentuje poniższy rysunek.

%rysunek trzba poprawić
\begin{figure}[h!]
  \centering
      \includegraphics[width=0.8\textwidth]{./obrazki/rozmieszczenie_budynkow.jpeg}
  \caption{Wzajemne rozmieszczenie budynków.}
\end{figure}



\chapter{Analiza potrzeb użytkowników – wymagania zamawiającego}
Na podstawie danych dostarczonych przez administratora obecnej sieci powstała analiza wykorzystania łącz. Przedstawiają ją poniższe tabele:

%tabele


\chapter{Założenia projektowe}
\chapter{Projekt sieci}
\section{Projekt logiczny sieci wraz z opisem koncepcji rozwiązania}
\section{Konfiguracja adresacji IP}
\section{Projekt okablowania}
\section{Projekt podłączenia do Internetu}
\section{Analiza bezpieczeństwa i niezawodności sieci}
\section{Kosztorys urządzeń}

%Otoczenie tabelki
%\begin{table}
 % \caption{Geograficzne rozmieszczenie odziałów firmy.}
 %\includegraphics[width=0.9\textwidth]{./obrazki/topologia.jpeg}
%\end{table}

%\begin{figure}[h!]
 % \centering
 %     \includegraphics[width=0.9\textwidth]{./obrazki/adres_computerbudy.jpeg}
%  \caption{Geograficzne rozmieszczenie odziałów firmy.}
%\end{figure}

%\chapter*{notatki}


\end{document}
