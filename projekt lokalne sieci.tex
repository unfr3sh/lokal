\documentclass{report}

\usepackage[a4paper, left=2.5cm, right=2cm, top=1.5cm, bottom=2cm, includefoot=false, includehead=false]{geometry}
\usepackage[MeX]{polski}
\usepackage[utf8]{inputenc}
\usepackage{enumerate}
\usepackage{graphicx}
\usepackage{float}
\usepackage{verbatim}

\makeatletter

\newcommand{\linia}{\rule{\linewidth}{0.4mm}}

\renewcommand{\maketitle}{\begin{titlepage}

    \vspace*{1cm}

    \begin{center}\small

    Politechnika Wrocławska\\

    Wydział Elektroniki W-4\\

    \end{center}

    \vspace{3cm}

    \noindent\linia

    \begin{center}

      \LARGE Projekt lokalne sieci komputerowe\\
      \normalsize\textsc{\@title}

         \end{center}

     \noindent\linia

    \vspace{0.5cm}

    \begin{flushright}

    \begin{minipage}{6cm}

    \textit{\small Autor:}\\

    \normalsize \textsc{\@author} \\

    \end{minipage}

    \vspace{5cm}

     {\small Prowadzący:}\\

         Dr hab. inż. Krzysztof Walkowiak

     \end{flushright}

    \vspace*{\stretch{6}}

    \begin{center}

    \@date

    \end{center}

  \end{titlepage}

}
\makeatother

\author{Mateusz Socha 181308 \\ Janusz Kuszczyński 184872 }

\title{}

\sloppy

\begin{document}

\maketitle
\tableofcontents

\chapter{Wstęp}

\vspace{0,5cm}
Projekt instalacji sieciowej jest realizowany dla firmy ComputerBudy. Siedziba która jest jednocześnie przedmiotem tego projektu znajduje się
przy ulicy Szwajcarska 22 w Wrocławiu.

\begin{figure}[H]
  \centering
      \includegraphics[width=0.8\textwidth]{./obrazki/adres_computerbudy.jpeg}
  \caption{Lokalizacja centrali firmy na mapie Wrocławia.}
\end{figure}

ComputerBudy jest firmą z działu IT. Zajmuje się ona zdalną pomocą przy problemach informatycznych. Zapewnia również zdalną administrację dla
skomplikowanych aplikacji na urządzeniach użytkownika. Jej oferta jest skierowana do osób prywatnych oraz małych i średnich firm, które
nie posiadają własnego działu IT.

Profil usług świadczonych powoduje, że brak połączenia z zewnętrzną siecią internet całkowicie paraliżuje całą firmę. 
Nawet awaria pojedynczego stanowiska powoduje straty. Restrykcyjna polityka bezpieczeństwa firmy sprawia, że nawiązanie połączenia z klientem
może nastąpić tylko z sieci firmowej. Aby zwiększyć bezpieczeństwo każde stanowisko obsługujące klientów jest przyłączone do sieci
za połączone za pomocą kabla UTP. Obostrzenia te spowodowane są obawą przed podsłuchaniem poufnych informacji przez osoby niepowołane oraz
przejęciem kontroli nad komputerem klienta podszywając się pod pracownika firmy z innej lokacji.

ComputerBudy wynajmuje łącznie 5 pięter w dwóch bliźniaczych budynkach stojących obok siebie.W pierwszym dwa i w następnym budynku kolejne 3.
Pozostałe piętra wynajmują inne firmy.

Celem naszej pracy jest stworzenie projektu nowej instalacji teleinformatycznej na użytkowanych przez firmę piętrach w obu budynkach.
\pagebreak[4]

 Zakres projektu:
\begin{itemize}
\item{Inwentaryzacja sprzętu i infrastruktury dostępnej w przedsiębiorstwie}
\item{Analiza potrzeb użytkowników – wymagania zamawiającego}
\item{Założenia projektowe}
\item{Projekt sieci}
\subitem{Projekt logiczny sieci wraz z opisem koncepcji rozwiązania}
\subitem{Konfiguracja adresacji IP}
\subitem{Projekt okablowania}
\subitem{Projekt podłączenia do Internetu}
\subitem{Analiza bezpieczeństwa i niezawodności sieci}
\subitem{Kosztorys urządzeń}
 
\end{itemize}
Wnioskując z profilu usług firmy priorytetowe znaczenie podczas projektowania należy nadać niezawodności. Drugim w kolejności czynnikiem jest oczywiście
szeroko pojęte bezpieczeństwo. Wskazane jest również zapewnienie łatwej możliwości rozbudowy sieci w tym budynku na kolejne piętra.
Oczywiście jako, że zleceniodawca jest firmą prywatną należy zminimalizować koszty całego przedsięwzięcia.

Do stworzenia projektu instalacji teleinformatycznej zostaną użyte szczegółowe plany budynków udostępnione przez zleceniodawcę.
Wymagania użytkowników zostaną opracowane na podstawie danych przekazanych przez administratora IT firmy oraz poprzez konsultację
z samymi pracownikami. Przepustowości łącz w nowej instalacji zostaną oszacowane na podstawie danych z obecnie istniejącej sieci komputerowej.

\chapter{Inwentaryzacja sprzętu i infrastruktury dostępnej w przedsiębiorstwie}
Na podstawie udostępnionej dokumentacji oraz wizyt w budynku mieszczącym firmę opracowano zestawienie zasobów obecnie posiadanych przez firmę.

\paragraph{Instalacja siecowa}W obecnej architekturze sieciowej razem w obu budynkach znajduje się 290 gniazdek ethernetowych. Nie wszystkie są obecnie używane.
Całe obecna instalacja opera się na elementach z kategorii 3. Jest to wyraźnie przestarzała technologia. Starą instalacje należy zdemontować a 
odzyskane elementy sprzedać. Działania te ma wykonać firma instalacyjna.

\paragraph{Serwery w firmie}
W centrali znajduje się serwer realizujący usługę
bazy danych, serwera FTP oraz hostujący stronę internetową firmy. Serwer działa pod kontrolą systemu NetWare. Znajduje się on w
pomieszczeniu nr 11 w budynku A. Pokój ten jest specjalnie przystosowany, posiada oddzielną klimatyzacje oraz jest 
dobrze zabezpieczone przed niepowołanym fizycznym dostępem. Takie samo pomieszczenie znajduje się w budynku B i ma również nr 11. Obecnie nie jest
używane. Właśnie w tych dwóch pomieszczeniach będą znajdować się urządzenia sieciowe oraz szafy krosownicze.

\paragraph{Sprzęt}Wszystkie komputery PC oraz inne urządzenia przyłączone do sieci posiadają interfejsy sieciowe ethernet i spełniają wymagania niezbędne
do połączenia do nowej sieci. Nasz projekt nie obejmuje zakupu urządzeń końcowych dla użytkowników.

\paragraph{Programy} Spis programów używanych w firmie:
\begin{enumerate}
 \item system operacyjny Windows XP
 \item przeglądarka Firefox
 \item program pocztowy Thunderbird
 \item Skype dla firm
 \item edytor tekstu Microsoft Office
 \item klient NetWare
 \item ssh
 \item TeamViever
 \item program księgowo kadrowy Płatnik
\end{enumerate}

\paragraph{Opis budynków}Budynki w których firma ma swoją siedzibę to nowoczesne biurowce. Wynajmujący piętro sam zagospodarowuje większość 
znajdującej się tam przestrzeni za pomocą modułowej architektury boxów. Aby umożliwić dużą elastyczność konfiguracji przestrzennej piętra 
wyposażone są w podwieszane sufity w których poprowadzono jest większość instalacji. Właśnie pod kątem tego montażu zostanie zaprojektowany plan
okablowania.

\paragraph{Zasilanie}Sieć energetyczna zainstalowana w budynku spełnia wszelkie wymagania dotyczące bezpieczeństwa oraz wydajności wymaganej dla sieci komputerowej.
Warte odnotowania jest obecność instalacji piorunochronowej na obu budynkach. Znacząco zwiększa to bezpieczeństwo sprzętów elektronicznych zainstalowanych
w budynku.

\paragraph{Zakłócenia}Zakłócenia elektromagnetyczne w budynku są na tyle małe, że można je pominąć. W okolicy nie pracuje żaden duży zakład przemysłowy, który mógłby 
znacząco wpłynąć na parametry zasilania w sieci. Inne firmy, które prowadzą swoją działalność w tych budynkach korzystają jedynie z standardowego
sprzętu biurowego połączonego kablową siecią ethernetową. Brak innych sieci bezprzewodowych w budynkach znacząco ułatwia implementacje sieci wifi
ponieważ nie występuje problem interferencji międzykanałowych.

\begin{comment}
\pagebreak[4]
\section{Plany budynków oraz ich wzajemne rozmieszczenie}
%Plany budynków
\begin{figure}[H]
  \centering
      \includegraphics[width=\textwidth]{./obrazki/plany_wew/a0.png}
\end{figure}

\begin{figure}[H]
  \centering
      \includegraphics[width=\textwidth]{./obrazki/plany_wew/a1.png}
\end{figure}

\begin{figure}[H]
  \centering
      \includegraphics[width=\textwidth]{./obrazki/plany_wew/b0.png}
\end{figure}

\begin{figure}[H]
  \centering
      \includegraphics[width=\textwidth]{./obrazki/plany_wew/b1.png}
\end{figure}

\begin{figure}[H]
  \centering
      \includegraphics[width=\textwidth]{./obrazki/plany_wew/b2.png}
\end{figure}
\end{comment}


%wzajemne rozmieszczenie budynków

\begin{figure}[H]
  \centering
      \includegraphics[width=0.5\textwidth]{./obrazki/rzut_terenowy.png}
    \caption{Wzajemne rozmieszczenie budynków na mapie.}
\end{figure}

\begin{figure}[H]
  \centering
      \includegraphics[width=0.8\textwidth]{./obrazki/rozmieszczenie_budynkow.jpeg}
    \caption{Wzajemne rozmieszczenie schemat.}
\end{figure}



\chapter{Analiza potrzeb użytkowników – wymagania zamawiającego}
Na podstawie danych dostarczonych przez firmowego administratora sieci sporządzono analizie ruchu sieciowego jaki wytwarzają
pracownicy w ciągu dnia roboczego. Przedstawiają je tabela 
\ref{tab:analiza dzialy} oraz \ref{tab:analiza podsumowanie}.

\begin{table}[H]

\centering
\caption{Analiza ruchu sieciowego w poszczególnych departamentach.
Tabele reprezentują ilość danych wygenerowanych przez 1 użytkownika danego departametu w ciągu dnia pracy. \label{tab:analiza dzialy}}

\includegraphics[width=0.5\textwidth]{./obrazki/ruch_tabele/d_ok.png}
\includegraphics[width=0.5\textwidth]{./obrazki/ruch_tabele/d_k.png}
\includegraphics[width=0.5\textwidth]{./obrazki/ruch_tabele/d_kad.png}
\includegraphics[width=0.5\textwidth]{./obrazki/ruch_tabele/d_b.png}
\includegraphics[width=0.5\textwidth]{./obrazki/ruch_tabele/d_m.png}   

\end{table}


\begin{table}[H]

\centering
\caption{Podsumowanie generowanego ruchu. \label{tab:analiza podsumowanie}}

\includegraphics[width=0.6\textwidth]{./obrazki/ruch_tabele/budynki.png}

\hspace{0,5cm}

      \includegraphics[width=0.6\textwidth]{./obrazki/ruch_tabele/podsumowanie.png}     
 
\end{table}




Wszyscy użytkownicy sieci korzystają z następujących programów: klient NetWare,Thunderbird, Firefox, Microsoft Office oraz Skype. Dodatkowo występuje 
oprogramowanie specjalistyczne dla wyszczególnionych działów. Księgowość i kadry pracują dużo na Płatniku natomiast obsługa klienta używa programu
do zdalnego zarządzania innymi komputerami TeamViever oraz ssh.

\paragraph{Plany rozwoju}Firma w najbliższym czasie planuje zakup dodatkowych 5 drukarek sieciowych.Po jednej dodatkowej na każde piętro. Drukarki te mają znajdować się
w ogólnie dostępnym miejscu.

\paragraph{Poufność danych}ComputerBudy w swojej bazie danych posiada nie tylko dane personalne swoich pracowników ale również klucze kryptograficzne wymagane do połączenia zdalnego
z maszyną klienta. Dane te są newralgiczne dla firmy. W związku z tym trzeba będzie w sieci koniecznie zastosować urządzenie typu firewall.

\paragraph{Obciążenie serwera www}Serwer na którym jest umieszczona strona firmy do poprawnego obsługiwania zapytań potrzebuje łącze o przepustowości 0,5 Mb/s do pobierania oraz
1 Mb/s do wysyłania. Wartości te będą uwzględnione dla wymagań dotyczących łącza internetowego.

\paragraph{Backup}Codziennie od godziny 24:00 do 6:00 rano wykonywany jest backup bazy danych klientów. Aby został poprawnie wykonany wymagana jest przepustowość 
na poziomie 3 Mb/s. Ponieważ czynność ta wykonywana jest w nocy wymaganie to będzie na pewno spełnione gdyż pracownicy nie będą generować ruchu
sieciowego.

\paragraph{Sieć wifi}Klient wyraził zapotrzebowanie na instalację sieci wifi dla działu obsługującego przedsiębiorców. W sali konferencyjnej często dochodzi  
do spotkań z klientami oraz małych narad zarządu. Wygodny dostęp dla internetu na pewno byłby czynnikiem ułatwiającym wszelkie negocjacje.
Niestety nie można przewidzieć zapotrzebowania na pasmo dla tego elementu sieci ponieważ nie wiadomo jaki program zechce uruchomić użytkownik.
Nie jest to obciążenie ciągłe sieci więc odpowiedni zapas przepustowości powinien rozwiązać ten problem.

Na podstawie zebranych danych można postawić wymagania dotyczące przepustowości sieci lokalnej oraz łącza z internetowego. Tabela \ref{tab:parametry_loncza}
prezentuje wymagania minimalne oraz zalecane. Wymagania minimalne zawierają wartości parametrów niezbędnych do poprawnego działania sieci. 
Niestety gdyby ich użyć mogłyby wystąpić problemy z jakością usług gdyby jakiś program przeciążył sieć. Aby tego uniknąć należy użyć wartości
zalecanych, które stanowią trzykrotność wartości minimalnej. Z takim zapasem przepustowości sieć będzie odporna na większość przeciążeń.


\begin{table}[H]
\caption{Przepustowości łącza internetowego. }
\label{tab:parametry_loncza}
 \centering
      \includegraphics[width=0.8\textwidth]{./obrazki/ruch_tabele/loncze.png}
\end{table}


\chapter{Założenia projektowe}

Na podstawie analizy potrzeb ComputerBudy, proponujemy
następujące rozwiązania:

\begin{itemize}

\item{technologia Gigabit Ethernet wykorzystana w okablowaniu pionowym}

\item{technologia Fast Ethernet wykorzystana w okablowaniu poziomym}

\item{symetryczne łącze z dostępem do Internetu o przepustowości 40 Mb/s}

\item{strukturę sieci oddzielającą serwery lokalne od zewnętrznych:
 \begin{itemize}
\item serwer WWW,
\item serwer Intranet,
\item serwer bazy danych,
\end{itemize}}

\item {użycie kabla UTP z kategorii 6,}

\item {urządzenia kompatybilne z IPv6 (router’y, switch’e, serwery),}

\item {urządzenia obsługujące technologię QoS}

\item{technologię VLAN w celu odseparowania jednostek organizacyjnych firmy}

\item{bezprzewodowy dostęp do sieci w dziale obługi klienta biznesowego w technologii WiFi
802.11n na częstotliwości 2,4Ghz}

\item{bezpieczeństwo sieci zapewnione sprzętowym firewall’em}

\item{dodatkowo ochrona realizowana przez specyfikę technologii VLAN,}

\item{redundantność oraz STP - w celu zapewnienia większej niezawodności sieci,}

\item{skalowalność dzięki zhierarchizowanemu podziałowi warstw.}

\end{itemize}


\chapter{Projekt sieci}
\section{Projekt logiczny sieci wraz z opisem koncepcji rozwiązania}
\section{Konfiguracja adresacji IP}
\section{Projekt okablowania}
\section{Projekt podłączenia do Internetu}
\section{Analiza bezpieczeństwa i niezawodności sieci}
\section{Kosztorys urządzeń}

%Otoczenie tabelki
%\begin{table}
 % \caption{Geograficzne rozmieszczenie odziałów firmy.}
 %\includegraphics[width=0.9\textwidth]{./obrazki/topologia.jpeg}
%\end{table}

%\begin{figure}[h!]
 % \centering
 %     \includegraphics[width=0.9\textwidth]{./obrazki/adres_computerbudy.jpeg}
%  \caption{Geograficzne rozmieszczenie odziałów firmy.}
%\end{figure}

\chapter*{notatki}

\begin{itemize}

\item sprawdzić z check-list -dodałem paragrafy aby się dobrze czytało
\item napisać,że ruch na 1 usera, 1 stanowisko, 1 godzine - gdzieś w okolicach tabelki - dodałem w opisie tabeli z wyliczeniami
\item plany budynków to nie inwentaryzacja, tylko planowana liczba gniazdek na podstawie info od admina
\item i to teraz jest zostanie zerwane/zniszczone bo kompletnie nie odpowiada potrzebom firmy- napisane w paragrafie
\item coś mówił że chciał ruch na serwer-napisane w paragrafie o serwerach 
\item "żeby było jasne że projekt nie obejmuje kupna komputreów"- napisane w paragrafie
\item "piszcie żeby było jak najmniej tego przebijania"- Przebijania scian?? Czy chodziło o tekst?? Planuje teraz pisać w paragrafach.
\item "napisać wprost w tabeli na ile to łącze"- Jak to kurwa na ile?? Ile lat??
\item "czy jest szacowanie ruchu na serwer ( a my że w tekście to jest :D)"- jest w paragrafie o serwerach 
\item "w inwentazyjacji opisujemy stan obecny"-tak dlatego wywalamy plany
\item to w "" to mniej wiecej jego słowa


 
\end{itemize}



\end{document}
